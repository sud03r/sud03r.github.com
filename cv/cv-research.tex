\documentclass[12pt]{article}
\usepackage[a4paper, total={6in, 9in}]{geometry}
\usepackage{hyperref}
\hypersetup{
  colorlinks   = true, %Colours links instead of ugly boxes
  urlcolor     = blue, %Colour for external hyperlinks
}
\date{}
\pagenumbering{gobble}

\begin{document}

\noindent{Neeraj Kumar \hfill +1-519-781-6026} \\
\noindent{\href{http://sud03r.github.com}{sud03r.github.com} \hfill \href{mailto:neeraj.kumar@uwaterloo.ca}{neeraj.kumar@uwaterloo.ca}\\
\rule{\textwidth}{0.1pt}

\paragraph{Research Interests} Algorithmic graph theory, Computational geometry, Algorithm Engineering.

\paragraph{Education}
\begin{itemize}
	\item \textbf{University of Waterloo, Canada}\\
		Master of Mathematics in Computer Science \textit{(Fall 2013 - current)}\\
		CGPA : $93.2\%$

	\item \textbf{Indian Institute of Technology, Varanasi, India}\\
		Bachelor of Technology in Computer Science and Engineering \textit{(2006 - 2010)}\\
		CGPA  : $8.69/10$
\end{itemize}

\noindent\textbf{Work Experience}
\begin{itemize}

	\item \textbf{Graduate Research Assistant, University of Waterloo}, Canada. \textit{(Sep 2013 - current)}\\
	Working on applying graph-theoretic concepts to software technology and verification of safety-critical systems.
	Currently, I am exploring DAG-width of structured programs and its application to $\mu$-caluculs model checking problem.

	\item \textbf{Senior Member Technical Staff, Mentor Graphics}, Noida, India. \textit{(July 2010 - Aug 2013)}\\
	Worked with Veloce compiler team on challenging optimization problems like partitioning, placement of large electronic designs.

\end{itemize}

\noindent\textbf{Selected Projects}
\begin{itemize}
	\item \textbf{Computation of Treewidth} \textit{Google Summer of Code 2014, OGDF}
	\href{http://www.google-melange.com/gsoc/project/details/google/gsoc2014/sudo\_er/5717271485874176}{[1]}\\
	Treewidth is a metric to measure tree-likeness of a graph. The goal of the project was to implement some heuristics
	for computing treewidth and tree-decomposition of a graph and efficiently solve some NP-hard problems on graphs of
	bounded treewidth.
	\item \textbf{Shared Libraries on NUMA} \textit{Course Project} \\
	In this project, we performed a holistic analysis of shared library performance on NUMA architectures. We wrote an
	evaluation paper\href{http://sud03r.github.io/numa.pdf}{[2]} summarizing our observations.
	
	\item \textbf{A test framework for scummVM's subsystems } \textit{Google Summer of Code 2010, ScummVM}
	\href{http://www.google-melange.com/gsoc/project/details/google/gsoc2010/sudo\_er/5757334940811264}{[3]}\\
	The objective of this project was to enhance the ScummVM unit testing infrastructure by implementing a Game Engine that
	could could invoke and test various ScummVM subsystems in an integrated and non-isolated manner.
\end{itemize}

\noindent\textbf{Technical Skills}
\begin{itemize}
	\item \textbf{Programming languages}: C++(Proficient), C (Good), Perl (Good), shell-scripts (Good), php/java (basic)
	\item \textbf{Operating systems}:  Linux (Ubuntu evangelist), Windows
	\item \textbf{Programming Tools}: GDB (Proficient), version control (git, svn, cvs), awk, sed, etc.  
\end{itemize}
\noindent\textbf{Miscellaneous}
\begin{itemize}
	\item \textbf{Publication} SiPTA: Signal Processing for Trace-based Anomaly Detection, at \textit{EMSOFT}'14
	\item \textbf{Scholarships} Graduate Entrance Scholarship (University of Waterloo), CBSE Merit Scholarship (India)
	\item \textbf{Opensource Software} Code contributions for OGDF, ScummVM and ES operating system.
	\item \textbf{Teaching Assistant} for CS341 : Algorithms; CS350 : Operating Systems and CS230 : Introduction to Computer Systems, at
		the University of Waterloo.
\end{itemize}

\end{document}
